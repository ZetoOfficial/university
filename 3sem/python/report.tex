\documentclass[a4paper,12pt]{article} %размер бумаги устанавливаем А4, шрифт 12пунктов
\usepackage[T2A]{fontenc}
\usepackage[utf8]{inputenc}
\usepackage[english,russian]{babel}%используем русский и английский языки с переносами
\usepackage[dvips]{graphicx} %вставка графики
\usepackage{pythonhighlight}
\usepackage{indentfirst}
\usepackage{geometry} % Меняем поля страницы
\usepackage{hyperref}
\usepackage{csquotes}

\hypersetup{
    colorlinks,
    citecolor=black,
    filecolor=black,
    linkcolor=black,
    urlcolor=black
}
\geometry{left=2cm}% левое поле
\geometry{right=2cm}% правое поле
\geometry{top=1cm}% верхнее поле
\geometry{bottom=2cm}% нижнее поле

\addto\captionsrussian{
	\renewcommand{\partname}{Глава}
	\renewcommand{\contentsname}{\centerline{Содержание}}
}

\begin{document}

\begin{titlepage}

    \begin{center}
        \large МИНИСТЕРСТВО НАУКИ И ВЫСШЕГО ОБРАЗОВАНИЯ РОССИЙСКОЙ ФЕДЕРАЦИИ \\
        \large «ТЮМЕНСКИЙ ГОСУДАРСТВЕННЫЙ УНИВЕРСТИТЕТ» \\
        \large ФИНАНСОВО-ЭКОНОМИЧЕСКИЙ ИНСТИТУТ \\[6cm]
        
        \large ОТЧЕТ ПО ИНДИВИДУАЛЬНОЙ РАБОТЕ \\[0.5cm]
        \large НА ТЕМУ «Парсинг информации о студентах» \\[5.1cm]
    \end{center}

    \begin{flushright} % выравнивание по правому краю
        \begin{minipage}{0.40\textwidth} % врезка в половину ширины текста
            \begin{flushleft} % выровнять её содержимое по левому краю

                \large \textbf{Работу выполнил:}\\
                \large Титов Павел Сергеевич \\
                \large {Группа:} ЛБ-01 \\

                \large \textbf{Преподаватель:}\\
                \large Актаев Нуркен Ерболатович

            \end{flushleft}
        \end{minipage}
    \end{flushright}

    \vfill

    \begin{center}
        \large Тюмень, \the\year
    \end{center}

\end{titlepage}

\tableofcontents
\newpage

\section*{Постановка задачи}
\addcontentsline{toc}{section}{Постановка задачи}
Реализовать парсер студентов ТюмГУ и выгрузку полученных данных в csv для анализа среднего балла и формы обучения.\\
Результат: .csv файл со списком студентов. \\
У каждого студента выделены следующие поля: 
\begin{enumerate}
    \item Уровень образования (Бакалавр/Магистр)
    \item Код специальности
    \item ФИО
    \item Рейтинг (Средний балл)
    \item Год поступления
\end{enumerate}
\newpage

\section{Листинг кода работы с парсером и выгрузки}

\textbf{Импорты}\\
Для обработки данных, их необходимо:
\begin{enumerate}
    \item Получить (UtmnParser)
    \item Обработать (json - Для выгрузки, кеширования и прочего, pydantic - для создания модели студента, tqdm - красивый progress bar)
    \item Сохранить/Выгрузить (csv)
\end{enumerate}

\begin{python}
import csv
import json

from pydantic import BaseModel
from tqdm import tqdm

from app.utmn_parser import UtmnParser
from settings import getLogger, settings
\end{python}

\textbf{Логгирование} \\
Инициализируем в начале файла логгер, чтобы в логах было понятно где выброшено исключение:
\begin{python}
logger = getLogger(__name__)
\end{python}

\textbf{Исключения} \\
Собственные классы ошибок позволят точнее определять, что пошло не так. 
\begin{python}
class InvalidUsername(Exception):
    pass


class StudentNotFound(Exception):
    pass
\end{python}


\textbf{Модели} \\
Модель студента для удобного представления данных:
\begin{python}
class Student(BaseModel):
    education_level: str
    specialty_code: str
    entered: int
    fio: str
    rating: float
    entered_upon: str
\end{python}


\textbf{Валидация} \\
Валидация позволит исключить студентов, не имеющих важные для сбора информации поля.
\begin{python}
def validate_student(student: dict, cached_students: dict):
    username = student['username']
    if username is None:
        raise InvalidUsername

    student_info = cached_students[username]
    if student_info is None:
        raise StudentNotFound
\end{python}


\textbf{Получение нужной информации о студентах} \\
Получение информации делится на несколько этапов: 
\begin{enumerate}
    \item Валидация
    \item Получение первичной информации
    \item Получение более подробной информации (данные из зачётной книжки)
    \item Компоновка данных в модель
\end{enumerate}
\begin{python}
def get_students_information(students: list, cached_students: dict):
    for student in students:
        try:
            validate_student(student, cached_students)
        except InvalidUsername:
            logger.error(f'{student} have invalid username')
            continue
        except StudentNotFound:
            logger.error(f'{student} not found')
            continue

        username: str = student['username']
        fio: str = student['displayName']
        student_info = cached_students[username]

        rating: float = 0.00
        entered_upon: str = 'No info'
        entered: int = student_info['main'].get('entered')
        specialty_code: str = student_info['main'].get('specialtyCode')
        education_level: str = student_info['main'].get('educationLevel')

        for studbook in student_info['studbooks']:
            if not studbook.get('active'):
                continue
            rating = student_info.get('rating').get('progress')
            if isinstance(rating, dict):
                rating = rating.get(student_info['defaultStudbook'], 0.00)
            entered_upon = studbook.get('enteredUpon')

        yield Student(
            education_level=education_level,
            specialty_code=specialty_code,
            entered=entered,
            fio=fio,
            rating=round(float(rating), 2),
            entered_upon=entered_upon,
        )
\end{python}


\textbf{Кэширование данных} \\
Чтобы каждый раз не запрашивать подробную информацию о студенте, создадим функцию сохранения. \\
Принцип работы следующий: 
\begin{enumerate}
    \item Получение уже сохранённой информации
    \item \enquote{Пробег} по всем студентам (с общей информацией)
    \item Сохранение, если подробных данных нет 
    \item Возврат сохранённых студентов
\end{enumerate}
\begin{python}
def cache_students(up: UtmnParser, students: list) -> dict:
    with open('cached_students.json', 'r', encoding='utf8') as f:
        cached_students = json.load(f)

    students_bar = tqdm(desc='Collecting student data', total=len(students))
    for student in students:
        username = student.get('username')
        if not cached_students.get(username):
            student_info = up.get_student(username).get('response')
            cached_students.update({username: student_info})
            with open('cached_students.json', 'w', encoding='utf8') as f:
                json.dump(cached_students, f, ensure_ascii=False)
        students_bar.update(1)
    students_bar.close()

    return cached_students
\end{python}


\textbf{Главная функция} \\
Собираем всё воедино:
\begin{enumerate}
    \item Создание экземпляра парсера
    \item Получение общей информации о студентах направления
    \item Получение подробной информации о студентах (из кэша, например)
    \item Сортировка + выгрузка данных в файл 
\end{enumerate}
\begin{python}
def main(study_plan: str, qualification: str, entered: int):
    up = UtmnParser(settings.app.usernameOrEmail, settings.app.password)

    students = up.get_all_students_by_study_plan(
        study_plan=study_plan,
        qualification=qualification,
        entered=entered,
    )

    cached_students = cache_students(up, students)

    students = sorted(
        get_students_information(students, cached_students),
        key=lambda student: (
            student.rating,
            student.education_level,
            student.specialty_code,
            student.entered,
        ),
        reverse=True,
    )
    fieldnames = list(Student.__fields__.keys())
    with open('students.csv', 'w') as f:
        writer = csv.DictWriter(f, fieldnames=fieldnames)
        writer.writeheader()
        for student in students:
            writer.writerow(student.dict())
\end{python}

\textbf{Запуск}
\begin{python}
if __name__ == '__main__':
    study_plan = 'students plan ...'
    qualification = 'qualification level ...'
    entered = 2021
    main(study_plan, qualification, entered)
    
\end{python}

\newpage

\section{Листинг кода парсера}
\textbf{Импорты}
\begin{enumerate}
    \item urlencode - Форматировать параметров для отправки
    \item requests - Для запросов к api
\end{enumerate}
\begin{python}
import json
from urllib.parse import urlencode

import requests
from settings import getLogger
from tqdm import tqdm
\end{python}


\textbf{Логгер}
\begin{python}
logger = getLogger(__name__)
\end{python}


\textbf{Исключения}
\begin{python}
class InvalidTokenException(Exception):
    pass
\end{python}


\textbf{Создания класса парсера}\\
В headers прописывается Content-Type для получения ответа в json формате, и User-Agent для доступа к api.\\
В init методе производится авторизация.
\begin{python}
class UtmnParser:
    _api_url: str = 'https://nova.utmn.ru/api/v1'
    _headers: dict = {
        'Content-Type': 'application/json',
        'User-Agent': 'Mozilla/5.0 (Macintosh; Intel Mac OS X 10_11_5) AppleWebKit/537.36 (KHTML, like Gecko) Chrome/50.0.2661.102 Safari/537.36',
    }

    def __init__(self, username: str, password: str) -> None:
        self._headers.update({'Authorization': self._get_token(username, password)})
\end{python}


\textbf{Метод получения токена для авторизации} \\
Метод авторизации работает следующим образом:
\begin{enumerate}
    \item Формирование данных для отправки (логин+пароль)
    \item Отправка запроса на авторизацию
    \item Получение токена
\end{enumerate}
\begin{python}
    def _get_token(self, username: str, password: str) -> str:
        '''Getting token by username and password

        Args:
            username (str): Username or e-mail
            password (str): Password

        Raises:
            InvalidTokenException: Authorization data is not valid

        Returns:
            str: Received token
        '''
        payload = json.dumps({'usernameOrEmail': username, 'password': password})
        resp = requests.post(
            f'{self._api_url}/auth/signin', headers=self._headers, data=payload
        )
        resp = resp.json().get('response')
        if token := resp.get('token'):
            return token
        raise InvalidTokenException()
\end{python}

\textbf{Метод получения всех студентов направления} \\
Получение делится на два этапа:
\begin{enumerate}
    \item Отправка запроса для получения информации о кол-ве студентов и страниц.
    \item Отправка запроса для получения всех студентов 
\end{enumerate}
Далее просто сбор полученных данных.
\begin{python}
    def get_all_students_by_study_plan(
        self,
        study_plan: str,
        qualification: str,
        entered: int,
    ) -> list:
        '''Getting all students of a given direction

        Args:
            study_plan (str): Full name of the study plan
            qualification (str): Qualification level
            entered (int, optional): Entered Year. The default is 2021.

        Returns:
            dict: query result
        '''
        params = {
            'limit': 1,
            'searchRole': 'student',
            'entered': entered,
            'studyPlan': study_plan,
            'offset': 0,
            'qualification': qualification,
        }
        limit = 20
        resp = requests.get(
            f'{self._api_url}/users?{urlencode(params)}', headers=self._headers
        ).json()
        total = resp.get('response').get('total')

        all_students = [*resp.get('response').get('users')]
        params['limit'] = limit
        students_bar = tqdm(desc='Collecting primary data on students', total=total)

        for offset in range(1, total + 1, limit):
            params['offset'] = offset
            resp = requests.get(
                f'{self._api_url}/users?{urlencode(params)}', headers=self._headers
            ).json()
            logger.debug(resp)
            all_students += resp.get('response').get('users')
            students_bar.update(limit)
        students_bar.close()

        return all_students
\end{python}

\textbf{Метод получения подробной информации} \\
Поскольку метод на получение информации о студентах направления возвращает общую информацию, необходимо запросить подробную информацию отдельным методом.
\begin{python}
    def get_student(self, username: str) -> dict:
        '''Getting detailed information about the student

        Args:
            username (str): student username on vmeste

        Returns:
            dict: query result
        '''
        return requests.get(
            f'{self._api_url}/users/username/{username}', headers=self._headers
        ).json()
\end{python}

\section{Листинг кода настроек}

\textbf{Настройки приложения:} \\
Логгирование, Модель настроек (данные для авторизации)
\begin{python}
from logging import INFO, FileHandler, StreamHandler, basicConfig, getLogger
from pathlib import Path

from pydantic import BaseModel, BaseSettings
from yaml import SafeLoader, load

CONFIG_FILE = str(Path(__file__).parent.absolute()) + '/settings.yaml'
LOGFILE_FILE = str(Path(__file__).parent.absolute()) + '/utmn.log'
basicConfig(
    level=INFO,
    format='[%(asctime)s] [%(levelname)s] [%(name)s] [%(funcName)s():%(lineno)s] %(message)s',
    handlers=[FileHandler(LOGFILE_FILE), StreamHandler()],
)
with open(CONFIG_FILE, 'r') as f:
    cfg = load(f, SafeLoader)


class App(BaseModel):
    usernameOrEmail: str
    password: str


class Settings(BaseSettings):
    app: App


settings = Settings.parse_obj(cfg)
\end{python}

\section{Файлы настроек и зависимостей}

\textbf{Файл конфигурации}
\begin{python}
app:
  usernameOrEmail: ''
  password: ''
\end{python}
\textbf{Файл с зависимостями для poetry}
\begin{python}
python = '^3.10'
flake8 = '^5.0.4'
pydantic = '^1.9.1'
pyyaml = '^6.0'
requests = '^2.28.1'
tqdm = '^4.64.0'
\end{python}
\newpage

\section*{Результат}
\addcontentsline{toc}{section}{Результат}

В результате получаем .csv файл с интересующей нас информацией:
\begin{table}[!ht]
    \centering
    \begin{tabular}{|l|l|l|l|l|l|}
    \hline
        education\_level & specialty\_code & entered & fio & rating & entered\_upon \\ \hline
        Бакалавр & 02.03.03 & 2021 & student fio & 5.0 & Бюджетная основа \\ \hline
        Бакалавр & 02.03.03 & 2021 & student fio & 4.71 & Бюджетная основа \\ \hline
        Бакалавр & 02.03.03 & 2021 & student fio & 4.14 & Бюджетная основа \\ \hline
        Бакалавр & 02.03.03 & 2021 & student fio & 3.29 & Полное возмещение затрат \\ \hline
        Бакалавр & 02.03.03 & 2021 & student fio & 0.0 & No info \\ \hline
    \end{tabular}
\end{table}

Полный код доступен на GitHub: https://github.com/ZetoOfficial/utmn-parser
\end{document}